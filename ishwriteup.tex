\documentclass{article}
\usepackage{graphicx}
\begin{document}

\begin{center}
{\Huge \textbf{Interactive Systems H}}\\
{\huge \textbf{Design Checkpoint}}\\
\vspace{5mm}
Dominic Small 2195530s\\
Michelle Dove 2142931d\\
Alistair Munn 2179147m\\
\end{center}

\section{Project Concept}
\label{sec:project}
Our project is a GPX viewing web app for runners who are looking to improve their fitness through ananlysis of data of previous runs from fitness trackers. It will allow them to look at previous times for a given route and how that has changed, as well as examining variables like the elevation change for that route. Runners can upload files and see the various stats from these runs, as well as share them with friends on social media.

It is also possible for runners to create routes for them to run using this app. This can give them information such as the time an average runner would take, giving them a goal to strive for. This information can also show a runner that they may not be ready for a run of this length, leading them to try a more approriate run. This average time can be compared to all their previous times in a graph, allowing the runners to see when they beat it.

Any skill level of runner could use this app, although it is aimed more at casual runners than professionals. However, anyone within this range (from a completely new runner to someone who does occaisional marathons) will find a use for it. Since it is aimed at helping runners improve themselves, those just starting out will probably find the most use for it, which will hopefully continue as they develop.

\section{User personas}
\label{sec:personas}
\underline{Bob's user profile}\\
Bob made a new years resolution to loose weight and has decided to do so by taking up jogging. To help monitor his performance he bought a wearable tracker and he would like to be able to get some basic feel-good stats, such as average distance run per day. Bob is not generally a tech fan and wants the interface to be as simple as possible.\\
%\vspace{5mm}
\underline{Janice's user persona}\\
Janice is a casual runner. She goes on runs regularly at the weekends to keep in shape and has been doing so for about a month now. Having noticed improvment in her fitness, she has decided to use an app to track her progress and see her improvement in more detail. She wants to see a record of her times to help descide if she should take on more strenuous runs or not.\\
In addition, Janice would like to compare her results to those of her friends to check how their fitness compares and motivate them (or her) to exercise more.\\
%\vspace{5mm}
\underline{Cameron's user persona}\\
Cameron is a very keen runner, and regularly trains. He has recently gotten into performance analysis and recording the data from his runs, which he monitors with a wearable device. Competing in several marathons, he is looking for an application to track all his statistics, and always wants to improve them. Most of all Cameron is interested in his speed and times over various distances, but is strongly interested in the other statistics obtained from the wearable device's data.\\

\section{Scenarios}
\label{sec:scenario}
\underline{Bob's Scenario:}\\
Looking at average times/ checking goal achievement.
Bob is thinking about possible routes to take when out for his run. He opens up the GPX viewer in edit mode to see a map of the area and plots out several possible routes. Once satisfied he saves them.\\
Bob is considering going for a weekend run and wants to decide on a route, so he opens up the GPX viewer and looks through his saved routes. He picks a route and heads out.\\
%\vspace{5mm}
\underline{Janice's Scenario:}\\
After improving, Janice will want to show off to her friends on social media. As such, she shares her times and routes on Facebook and invites them to sign up so she can compare herself directly to them as they hopefully start running. When her friends join, Janice marks them as friends on the app so she can keep an eye on their statistics as well as her own.\\
Janice will also use the app’s ability to calculate average times for the routes she runs to set herself a goal. Once she beats this time consistently she may consider running more frequently and start using more advanced features of the app and of course, share this information with her friends.\\
%\vspace{5mm}
\underline{Cameron's Scenario:}\\
Cameron has just finished a 10 mile run and has just uploaded the data from his run. He is very interested to see his time for the run and to check his speed not only for the entire run but to see how this varied during different segments of the run. He is particularly interested to see how his speed varied for the more hilly miles, where there was a large change in elevation.\\
Cameron has completed another marathon in a record time of 3 hours and 45 minutes. He is very proud of this and wants to share his achievement with his friends. He wants them to be able to see all the data from his marathon, especially his time and speed.\\

\section{Wireframes}
\label{sec:wire}

\begin{figure}[!htbp]
\includegraphics[width=\linewidth]{"Wireframes/images/account creation".png}
\caption{Account Creation}
\end{figure}

\begin{figure}[!htbp]
\includegraphics[width=\linewidth]{"Wireframes/images/Home Page (creating route)".png}
\caption{Creating Route}
\end{figure}

\begin{figure}[!htbp]
\includegraphics[width=\linewidth]{"Wireframes/images/Home Page (uploading run)".png}
\caption{Uploading Run}
\end{figure}

\begin{figure}[!htbp]
\includegraphics[width=\linewidth]{"Wireframes/images/Home Page (viewing graphs)".png}
\caption{Viewing Graphs}
\end{figure}

\begin{figure}[!htbp]
\includegraphics[width=\linewidth]{"Wireframes/images/Home Page (viewing route)".png}
\caption{Viewing Route}
\end{figure}

\end{document}
